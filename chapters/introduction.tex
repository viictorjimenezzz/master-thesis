\chapter{Introduction}\label{sec:introduction}

This chapter aims to set the stage for the detailed analysis and discussion that will follow, by 
providing a general overview of the problem of model robustness in machine learning and the current
approaches to address it.

\section{Motivation and objectives}\label{sec:motivation}

- Introduce deep learning, the current development and the implications to society. \\
- Explain what is robustness, give examples and why is it critical for the development and implementation of DL.
Give the example of cow/human image classification (funny, switzerland)\\
- Besides the funny example, provide extensive argumentation to highlight why is solving it is of pararmount importance \\
- Outline (no maths) why is it a difficult problem. \\
\\
- Explain current approaches to robustness and ML development in general => trash accuracy. \\
- Introduction to posterior agreement as a robustness measure. Cite Joao+Alessandro paper, in which
the conditions for a robustness metric are outlined.  \\
- Is it necessary to go into the record of PA in other settings ? \\
- Show initial results from the paper, and explain why it is a promising approach.  \\
\\
- Lead to the derivative work that will be presented in the thesis.=> benchmarking \\
- Lead to derivative but next level (more useful, current interest...) work => model selection \\
- Lead to non-derivative (i.e. probably unsuccessful) work => model selection beyond robustness \\
- Outline the structure of the thesis in terms of hypothesis. \\


\section{Related work}\label{sec:structure}

- Adversarial learning \\
- Domain adaptation \\
- Model selection for robustness \\